\section{Abstract}
På enkelte fakulteter på Syddansk Universitet, udgiver underviserne en liste af tilgængelige emner når de studerende skal arbejde på projekter. Ud fra denne liste indgiver de studerende en prioriteret liste over hvilke emner de helst vil arbejde med, hvorefter emnerne bliver fordelt baseret på de studerendes præferencer. Denne fordeling af projekterne sker med et program kontrolleret via kommandolinien og som kun er tilgængeligt på en enkelt computer.\\
I denne rapport beskriver jeg arbejdet med at udvikle en web applikation hvis formål er at virke som en grafisk brugerflade til fordelings programmet, og gøre det tilgængeligt over internettet for at gøre brugen af programmet nemmere og mere tilgængelig.\\\\
I rapporten beskriver jeg funktionaliteten af fordelings programmet, hvorfor jeg valgte at bruge Django og SQLite som base i udviklingen af applikationen og giver eksempler på alternativer til disse.\\
Jeg gennemgår hvordan applikationen blev gjort tilgængelig over internettet ved hjælp af Gunicorn og NGINX, vejen fra kode til kørende applikation og hvordan applikationen blev testet .\\
Til slut beskriver jeg hvad der kan gøres for at gøre applikationen tilgængelig og brugbar i et miljø udenfor Syddansk Universitet og giver bud på funktioner der kan implementeres i fremtiden for at øge funktionaliteten og forbedre brugeroplevelsen af applikationen.