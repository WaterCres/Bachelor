\section{Testing}\label{test}
When developing any kind of software, testing the application is an important step in the process. Tests are performed on the application to discover any potential bugs in the software before the application is deployed and used in a setting in which a bug could potentially be extremely disruptive and/or make the entire application unusable.
\\Additionally tests in which users interact with the application can be performed. These tests allows developers and users to collaborate and identify the components that work, and if some things need to be done in another way.\\\\
During the development of this application both kinds of test were performed. User testing where the application was presented to a representative of the intended end user in order receive feedback on the user experience and suggestions on features that would make the application more appealing to use.\\
And technical tests whenever a new feature was implemented. The feature was tested by trying to break it, in order to either prevent or gracefully handle the failure of the feature. An example of this is when I tested the file uploading feature, several different file formats were uploaded as well as excel spreadsheets formatted in different ways. As a part of these tests a logging feature was created to write error messages to a file, in order to review what was attempted when the error happened.
\subsection{User testing}
Early in the process when the first prototype featuring the file upload and assignment pages was ready, a test session was arranged with an end user. The session started with a quick introduction to the application explaining its features, the current state and the scope of the test. The scope of this test was primarily to receive feedback on the navigation of the page, in addition to getting an impression of what features would be important to the user.\\
The outcome of the test was that the site was intuitive to navigate, it was however revealed that the format and content of the cells in the uploaded files would not necessarily be static or conform to the initially expected format. Furthermore some of the options on the assignment page could lead to confusion unless properly explained.\\\\
To solve these issues the data types saved to the database were reworked to better accommodate flexibility in the content of the cells of the uploaded spreadsheet, and an intermediate page was added on which the user is asked to match the headers of the uploaded file to the data needed to perform the computation. On the assignment page some of the options were hidden under an ``Advanced features'' menu, with an attached section explaining what each option does.\\\\
Later in the process another user test was arranged. At that point the application had all the features and components needed to upload documents, start the computation and review the results along with the ability to log in and restrict the functionality based on the permissions given to a user. The primary scope of this test was to verify that the login and permission system worked as intended, and secondly to verify that the application could be used to solve the allocation problem and thereby solving the problem this project was based on.\\
This second test showed that the login and permission systems worked as intended. An issue was discovered where the user was unable to match headers to data in the uploaded document, when the application was accessed with the Microsoft Edge browser, this issue was not present when the user accessed the application with Firefox. The user was unable to start the allocation process, this issue was later revealed to be caused by misconfigured javascript which disabled the button used to start the process.
\\Based on feedback from the user some UI elements were redesigned in order to better match the workflow.
