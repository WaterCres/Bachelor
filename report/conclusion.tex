\section{Conclusions and future work}\label{conclusion}
My goals for this project were to build an interface where authorized users would be able to upload spreadsheets containing student and topic data. The users had to be able to start the allocation process and view the results of this process. Additionally a tool had to be setup in order to facilitate quick and automatic updates.\\\\
While my application along with the running instance of Jenkins has all the features and functionality I set out to implement, there are some aspects which could and maybe should be changed in future updates, or at least should be considered if the application were to be hosted outside of the SDU environment.\\\\
At the moment the SQLite database file is stored within the docker container. As a consequence of this the entire database is purged whenever the docker image is rebuilt. Meaning the data stored in the database will be deleted when the application is updated. To remedy this two solutions exists. One solution is to store the database file outside the docker container by linking a directory outside of the docker container to a directory inside it, and setting the path to the database file accordingly in the Django settings. This solution does not require a lot of development time, it does however leave the database vulnerable to manipulation from other processes running on the server. Alternatively the database could be migrated from the current SQLite database to an instance of MySQL or PostgreSQL running in a separate container. This approach does require more development work, but allows multiple running instances of the application to access the same database in a controlled manner, even if these instances were distributed among several computers.\\\\
Currently the NGINX configuration is tailored to run on a virtual machine provided by SDU.\\
In order to have the application running and accessible on another computer in general, a self signed certificate should not be used, instead something like Certbot could be utilised to automate the certificate process. The host address specified in the NGINX configuration and \verb|.env| file will also have to be changed to match the new url address associated with the application. An alternative would be hosting the application on a service like \href{https://railway.app/}{Railway}, this approach leaves the configuration and handling of the reverse proxy, certificates and DNS records to the service provider and lets a developer focus on the application itself.\\\\
With regards to the login system on the application, while the transition to another Azure Active Directory is fairly simple, as I described in the \hyperref[sec:login]{login section}, transitioning to a completely different service does require a bit more work. The transition would require the removal of the Microsoft authentication module currently integrated into the application, and the setup of the new system along with potentially integrating a new module if the desired login system has a Django package.\\\\
Currently no automatic testing has been implemented in Jenkins or the unit test module included in Django. This should absolutely be remedied in  the future, in order to avoid the introduction of potentially application breaking bugs to the running instance of the application.\\
Unit tests of the individual functions in the application can be written as python scripts to utilise the included test module. In addition to unit tests, tests could be created using \href{https://www.selenium.dev/}{Selenium} in order to test the flow through the application and if the design and functionality is consistent between different browsers.\\
These tests should be executed by Jenkins before building and restarting the application.\\\\
In addition to the aspects outlined above, I would consider the following features to be great additions to the application in the future:\\
\begin{itemize}
	\item The creation of a custom admin page.\\
	In order to keep the UI on this page consistent with all the other pages. This would also allow administrators of the application to get a more detailed overview of the data currently stored in the database since the current admin page does not display the data in a way that suits the needs of the users very well.
	\item Associate student and topic data to the case in which they are involved.\\
	Currently when the allocation process is started all selected student types and all projects present in the database will be considered during the allocation and while this works fine as long as only one case is present in the database, it creates a problem in the future when the application must handle data from several different cases.
	\item Update the data submission page to allow more flexibility.\\
	This flexibility includes the option to overwrite/update existing data when a document is uploaded and matching entries are found, an option to manually enter a single entry in to the database from this page and allowing other file formats, in addition to \verb|*.xlsx|, to be uploaded.
	\item An option to extract data from the application in a suitable file format.\\
	At the moment all the data used in the application is only accessible through the application, the option to extract certain data e.g. the files created as output from the computation, was requested by a user involved in the user tests.
\end{itemize}
During my work on this project I have become aware of, and familiar with several tools and technologies used in the world of software development, and identified the tools I found most suitable for this project.\\Decisions made in the development process, may however need reconsideration if the requirements change.